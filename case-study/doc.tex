\documentclass[a4paper, titlepage,12pt]{article}
\usepackage[margin=3.7cm]{geometry}
\usepackage[utf8]{inputenc}
\usepackage[T1]{fontenc}
\usepackage[swedish,english]{babel}
\usepackage{csquotes}
\usepackage[hyphens]{url}
\usepackage{amsmath,amssymb,amsthm, amsfonts}
\usepackage[backend=biber,citestyle=ieee]{biblatex}
\usepackage[yyyymmdd]{datetime}
\usepackage{pgfgantt}

\addbibresource{literature.bib}

\title{Distributed Denial of Service (LOIC, HOIC)}
\author{Adam Temmel (adte1700)}

\begin{document}
	\maketitle
	\section{Introduction}\label{sec:introduction}
		A Distributed Denial of Service attack is a method of disturbing and, ultimately, denying a service. Such an attack attempts to overload the targeted service by sending a copious amount of requests from a variety of sources. There are several different approaches to exactly how the service is overloaded, with the common denominator being that the attack is performed by multiple actors all over the world, which greatly limits the possibilities of stopping the attack by blocking a single source. 
	\subsection{Usage}\label{sec:usage}
		The method has somewhat of a bad reputation, as it historically has been used for several different cybercrimes, in which one or multiple popular online services have been brought to a halt by targeting them with DDoS attacks. Common motivators for these attacks include, revenge, blackmail and various forms of activism. A recent example of such an attack occurred in June, 2019, where the popular messaging application Telegram was brought down in an attempt to prevent Hong Kong protesters  from communicating with each other. That said, this method is also used by companies to stress-test their own service, effectively simulating a system under heavy load.
	\subsection{Low Orbit Ion Cannon (LOIC)}
		Low Orbit Ion Cannon is the name of open source software built to perform DDoS attacks. It is nowadays considered to be largely ineffective, as security experts have stated most attacks that LOIC can offer are easily mitigated by a properly configured firewall. Historically, however, it has been used in enough attacks for improper usage of the tool to be recognized as a felony by the Computer Fraud and Abuse Act. The name Low Orbit Ion Cannon is believed to have been derived from the fictional weapon of the same name from the Command and Conquer video game series. This works as a suitable analogy to how the software, if given enough resources, operates as a point and click weapon, much like its video game counterpart.
	\subsection{High Orbit Ion Cannon (HOIC)}
		Created as an answer to the limitations of the LOIC, the High Orbit Ion Cannon (HOIC)
	\section{DDoS explanation}
	\subsection{Expected results}
	\section{Research method}
	%\subsection{Type of input(s)}
	\subsection{Related research 1}
	\subsection{Related research 2}
\end{document}
